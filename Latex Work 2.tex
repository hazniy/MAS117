\documentclass[11pt, a4paper]{amsart}

\usepackage{hyperref} %make it neat 
\begin{document} 

\section{Probability at the University of Sheffield}

\emph{``The Applied Probability Trust''}
Our members engage in a wide range of research topics, 
including
\begin{itemize}
	\item branching processes
	\item random walk
	\item random graphs
	\item random fractal
	\item stochastic processes
\end{itemize}

\subsection{Members}
The probability group consists of the following academic staff
\begin{itemize}
	\item Dr Nic Freeman
	\item Dr Jonathan Jordan
	\item Dr Bas Lodewijks
	\item Dr Rosie Shewell Brockway
	\item Dr Robin Stephenson
	\item Dr Mark Yarrow
	\item Dr Maksim Zhukovskii
\end{itemize}

The material here comes from the website \url{https://bbc.com}.
\end{document}.

%homework 
\documentclass[11pt, a4paper]{amsart}
\usepackage{parskip} 
\title{MAS116/117: Homework 2}
\author{Nur Farhazni Najwa Binti Farhan}

\begin{document}

\maketitle
\section{Mathematics and Statistics at the University of Sheffield}
There is a strong research profile in mathematics and statistics at the University of Sheffield. Below is a list of the research areas of the school:

\begin{itemize}
    \item Algebra and Algebraic Geometry
    \item Fluid Dynamics
    \item Gavitation and Cosmology
    \item Mathematical Bio/Environmental Dynamics
    \item Number Theory
    \item Probability
    \item Solar Physics and Plasma Dynamics 
    \item Statistics
    \item Topology 
\end{itemize}

\subsection{Dr Yi Li's Research Interests}
\emph{Dr Yi Li's main research interests are in the field of fluid mechanics, in particular turbulence. Topics include fluid flow optimisation, simulation and modelling, and the application of signal processing and database techniques in the study of data-intensive fluid mechanics problems.}

Turbulence, a key area in fluid mechanics, refers to chaotic fluid movement where pressure and flow speed change unpredictably. Unlike smooth, steady laminar flow, turbulent flow is full of swirl and eddies, as seen in wind and rivers. Understanding turbulence helps improve the design and function of systems like pumps, turbines, and aircraft. 

Turbulence focus on making fluid flow more efficient and creating better simulations to predict how fluids behave. Dr Yi Li use tools like signal processing and data analysis to handle the large amounts of data involved in fluid studies. This research helps in fields such as energy, environmental science, and engineering, where understanding and controlling fluid flow is crucial for better performance. 

{Sources:}
\begin{itemize}
    \item `Turbulence' Wikipedia The Free Encyclopedia
    \item `Turbulent flow' by The Editors of Encyclopaedia Britannica (Oct 11, 2024)
\end{itemize}

\section{Solution Rewrite}
\subsection{Question}
Find the equation of the line $L$ that passes through the points $A(8, 1)$ and $B(2, 3)$

\subsection{Solution}
First, we need to calculate the gradient $(m)$ of the equation. Based on the given information, we can find the gradient $(m)$ :

\begin{align*}
	m 
	&= \frac{y_2 - y_1}{x_2 - x_1} \\
	&= \frac{3 - 1}{2 - 8} \\
	&= \frac{2}{-6} \\
	&= -\frac{1}{3}.
\end{align*}

Then, we need to find the y-intercept $(c)$ of the equation. We can substitute one of the points, into the equation $y = mx + c$ to find the $c$.

\begin{align*}
	1 
	&= -\frac{1}{3}(8) + c \\
	&= -\frac{8}{3} + c \\
	c 
	&= 1 + \frac{8}{3} \\
	&= \frac{3}{3} + \frac{8}{3} \\
	&= \frac{11}{3}.
\end{align*}

Finally, we know $m =  -\frac{1}{3}$ and $c = \frac{11}{3}$. We can substitute both values into the straight line equation $y = mx + c$ which produces the equation of $L$ :

\[
 y = -\frac{1}{3}x + \frac{11}{3}.
\]

\end{document}




