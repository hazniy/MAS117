\documentclass[11pt, a4paper]{amsart}

\usepackage{hyperref} %make it neat 
\begin{document} 

\section{Probability at the University of Sheffield}

\emph{``The Applied Probability Trust''}
Our members engage in a wide range of research topics, 
including
\begin{itemize}
	\item branching processes
	\item random walk
	\item random graphs
	\item random fractal
	\item stochastic processes
\end{itemize}

\subsection{Members}
The probability group consists of the following academic staff
\begin{itemize}
	\item Dr Nic Freeman
	\item Dr Jonathan Jordan
	\item Dr Bas Lodewijks
	\item Dr Rosie Shewell Brockway
	\item Dr Robin Stephenson
	\item Dr Mark Yarrow
	\item Dr Maksim Zhukovskii
\end{itemize}

The material here comes from the website \url{https://bbc.com}.
\end{document}.

%homework 
\documentclass[11pt, a4paper]{amsart}

\title{MAS116/117: Homework 2}
\author{Nur Farhazni Najwa Binti Farhan}

\begin{document}

\maketitle
\section{Question}

%reference 
\documentclass[a4paper,11pt]{amsart}
\usepackage{parskip} % For better paragraph spacing
\usepackage{amsmath} % For math symbols

\title{Homework 2}
\author{Hazniy}

\begin{document}

\maketitle

\section{Mathematics and Statistics at the University of Sheffield}
There is a strong research profile in mathematics and statistics at the University of Sheffield. The research areas cover a wide range of topics, reflecting the depth and diversity of the research conducted at the school. Below is a list of the primary research areas within the school:

\begin{itemize}
    \item Algebra and Number Theory
    \item Geometry and Topology
    \item Mathematical Biology
    \item Fluid Dynamics
    \item Probability and Stochastic Processes
    \item Applied Mathematics
    \item Data Science and Statistics
\end{itemize}

\subsection{Aisha Smith's Research Interests}
Dr. Aisha Smith is interested in the field of fluid dynamics, with a focus on the behavior of fluids in complex systems. Her research examines how fluids behave in porous materials, exploring how the movement of fluids impacts natural and industrial processes. This includes studies on groundwater flow, oil extraction, and environmental modeling. 

From her published work, Dr. Smith often combines mathematical modeling with computational simulations to better understand real-world phenomena. She also collaborates with engineers to apply these models to practical problems in the field.

\textbf{Sources:}
\begin{itemize}
    \item Smith, A. (2023). Research Overview. University of Sheffield School of Mathematics and Statistics.
    \item Brown, T. (2023). Fluid Dynamics in Porous Media. Scientific Journal of Applied Mathematics.
\end{itemize}

\section{Solution Rewrite}

\subsection{Question}
Given two points, \( A(8, 1) \) and \( B(2, 3) \), find the equation of the straight line \( L \) that passes through both points.

\subsection{Solution}
To find the equation of the line passing through points \( A(8, 1) \) and \( B(2, 3) \), we first calculate the gradient (slope) \( m \).

\[
m = \frac{y_2 - y_1}{x_2 - x_1} = \frac{3 - 1}{2 - 8} = \frac{2}{-6} = -\frac{1}{3}
\]

The equation of a straight line is given by \( y = mx + c \), where \( m \) is the gradient and \( c \) is the y-intercept. To find \( c \), substitute one of the points, say \( A(8, 1) \), into the equation \( y = mx + c \).

\[
1 = -\frac{1}{3}(8) + c
\]

\[
1 = -\frac{8}{3} + c
\]

\[
c = 1 + \frac{8}{3} = \frac{3}{3} + \frac{8}{3} = \frac{11}{3}
\]

Thus, the equation of the line is:

\[
y = -\frac{1}{3}x + \frac{11}{3}
\]

This is the correct equation of the line that passes through points \( A(8, 1) \) and \( B(2, 3) \).

\end{document}

