\documentclass[11pt, a4paper]{amsart}
\usepackage{parskip} 
\usepackage{amssymb} 
\usepackage{amsthm} 
\newtheorem{thm}{Theorem}[section]
\newtheorem{lem}[thm]{Lemma}
\title{MAS116/117: Lab 3 Experiments}
\author{Nur Farhazni Najwa Binti Farhan}

\begin{document}

\maketitle
\section{Typesetting practice}
\begin{itemize}
    \item The formula for the addition of velocities in relativistic mechanics is 
   \[
   f(u, v) = \frac{u + v}{1 + \frac{uv}{c^2}}
   \]
    \item TeXmaker makes it easy to find Greek Letters and do typeset things like 
   \[
   \Xi^\Psi\Omega = \frac{\theta}{\gamma}.
   \]
   \item Pointing and clicking in TeXmaker also helps with things like  
   \[
   \underbrace{\circlearrowleft \ldots\circlearrowleft.} _ \text{n times}
   \]
   \item Typically we use $\mathbb{R}$ to denote the real numbers and $\mathbb{C}$ to denote the complex numbers.
\end{itemize}

\section{The square-root of 2}
We are going to investigate a solution of the equation
\begin{equation}
	x^2 = 2.
\label{eq:root-2} 
\end{equation}
The positive solution to equation\textasciitilde (\ref{eq:root-2}) is denoted $\sqrt{2}$. 

\begin{lem}
	Any rational number can be written in the form $a/b$ with $a$ and $b$ coprime integers.
\end{lem}
\begin{proof}
	Suppose that we have a rational number $p/q$ where $p$ and $q$ are integers with $q \neq 0$. Blah blah blah.
\end{proof}
\begin{thm}
	The real number $\sqrt{2}$ is irrational.
\end{thm}
\begin{proof}
	We prove this by contradiction. First we assume that $\sqrt{2}$ is rational and so can be written as $a/b$ for \emph{coprime} integers $a$ and $b$. Blah blah blah. 
\end{proof}
\end{document}
