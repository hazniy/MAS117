\documentclass[11pt, a4paper]{amsart}
\usepackage{parskip} 
\usepackage{amssymb} 
\usepackage{amsthm} 
\newtheorem{thm}{Theorem}[section]
\newtheorem{lem}[thm]{Lemma}
\title{MAS116/117: Lab 3 Experiments}
\author{Nur Farhazni Najwa Binti Farhan}

\begin{document}

\maketitle
\section{Typesetting practice}
\begin{itemize}
    \item The formula for the addition of velocities in relativistic mechanics is 
   \[
   f(u, v) = \frac{u + v}{1 + \frac{uv}{c^2}}
   \]
    \item TeXmaker makes it easy to find Greek Letters and do typeset things like 
   \[
   \Xi^\Psi\Omega = \frac{\theta}{\gamma}.
   \]
   \item Pointing and clicking in TeXmaker also helps with things like  
   \[
   \underbrace{\circlearrowleft \ldots\circlearrowleft.} _ \text{n times}
   \]
   \item Typically we use $\mathbb{R}$ to denote the real numbers and $\mathbb{C}$ to denote the complex numbers.
\end{itemize}

\section{The square-root of 2}
We are going to investigate a solution of the equation
\begin{equation}
	x^2 = 2.
\label{eq:root-2} 
\end{equation}
The positive solution to equation\textasciitilde (\ref{eq:root-2}) is denoted $\sqrt{2}$. 

\begin{lem}
	Any rational number can be written in the form $a/b$ with $a$ and $b$ coprime integers.
\end{lem}
\begin{proof}
	Suppose that we have a rational number $p/q$ where $p$ and $q$ are integers with $q \neq 0$. Blah blah blah.
\end{proof}
\begin{thm}
	The real number $\sqrt{2}$ is irrational.
\end{thm}
\begin{proof}
	We prove this by contradiction. First we assume that $\sqrt{2}$ is rational and so can be written as $a/b$ for \emph{coprime} integers $a$ and $b$. Blah blah blah. 
\end{proof}
\end{document}

%homework%
\documentclass[11pt, a4paper]{amsart}
\usepackage{parskip} 
\usepackage{amssymb} 
\usepackage{amsthm} 
\newtheorem{thm}{Theorem}[section]
\newtheorem{lem}[thm]{Lemma}
\newtheorem{defn}[thm]{Definition}
\title{MAS117: Homework 3}
\author{Nur Farhazni Najwa Binti Farhan}

\begin{document}

\maketitle

\section{The square-root of 2}
We are going to investigate a solution of the equation
\begin{equation}
	x^2 = 2.
\label{eq:root-2} 
\end{equation}

\begin{defn}
The positive solution to equation (\ref{eq:root-2}) is denoted $\sqrt{2}$. 
\end{defn}

\begin{lem}
	Any rational number can be written in the form $a/b$ with $a$ and $b$ coprime integers.
\end{lem}

\begin{proof}
	Two numbers $a$ and $b$ are said to be coprime if their greatest common divisor is 1. A rational number is any number that can be expressed as a fraction $a/b$, where $a$ and $b$ are integers (whole numbers) and $b$ is not zero. We can start with a rational number $r = \frac{a'}{b'}$ and find the greatest common divisor of $a'$ and $b'$, called $c$. The greatest common divisor is the largest number that divides both $a'$ and $b'$. We can simplify the fraction by dividing both $a'$ and $b'$ by $c$ which we will get equation $a = \frac{a'}{c}$ and $b = \frac{b'}{c}$. Now, $a$ and $b$ have no common factors so greatest common divisor is 1. The fraction $a/b$ is in its simplest form, and $a$ and $b$ are coprime. 
\end{proof}

\begin{thm}
	The real number $\sqrt{2}$ is irrational.
\end{thm}
\begin{proof}
	We prove this by contradiction. First we assume that $\sqrt{2}$ is rational and then show that this leads to something impossible. Let's assume that $\sqrt{2}$ can be written as a fraction $\frac{a}{b}$, where $a$ and $b$ are whole numbers and the fraction $\frac{a}{b}$ is in simplest form. We proceed with the following: 
\[
   \sqrt{2} = \frac{a}{b}.
\]
To eliminate the square root, we square both sides of the equation. 
\[
   (\sqrt{2})^2 = (\frac{a}{b})^2
\]
\[
 2 = \frac{a^2}{b^2}.
\]
After squaring, we multiply both sides by $b^2$ to clear the denominator. This gives us the following equation: 
\[
   2b^2 = a^2.
\]
This implies that $a^2$ is even because it is equal to $2b^2$, and any number multiplied by $2$ is even.  Since $a^2$ is even, $a$ must be even. Therefore, we can express $a$ as $a = 2c$, where $c$ is some whole number. Substitute $a = 2c$ into the equation: 
\[
   2b^2 = (2c)^2
\]
\[
   b^2 = 2c^2.
\]
This shows that $b^2$ is also even, implying that $b$ is even too. Now we've shown that both $a$ and $b$ are even, we conclude that they share a common factor of $2$, which contradicts our original assumption that $a$ and $b$ have no common factors.
\end{proof}

\end{document}

