\documentclass[11pt, a4paper]{amsart}
\usepackage{parskip} 
\usepackage{pgfplots}
\usepackage{graphicx}
\usepackage{hyperref} %produce clickable link
\usepackage{parskip} 
\usepackage{amssymb} 
\usepackage{amsthm} 
\usepackage{booktabs} 
	\newcommand{\otoprule}{\midrule[\heavyrulewidth]}
\title{MAS116/117: Lab 5 Experiments}
\author{Nur Farhazni Najwa Binti Farhan}

\begin{document} 
\maketitle
\tableofcontents
\section*{Recommended fiction} %* make the number more neat 
A friend recommended I read some books~\cite{Lessing:Notebook} by Doris Lessing, starting with one of her most famous books. I spent six hours straight just eating up the content of this book~\cite{Coho:iewu}. And I think it's safe to say those six hours changed my life.  This book was an EXTREMELY fast-paced read for me since I was on the edge of my seat the whole time. 
Wiles was the person who proved Fermat's Last Theorem~\cite[Theorem~2.1]{Wiles:fermats-theorem}. %specific page/chapt of book 

\begin{thebibliography}{99} %99 informs the widest number it is expecting for the citation label
	\bibitem{Lessing:Notebook}
		D.~Lessing,
		\emph{The Golden Notebook},
		HarperCollins Publishers, 2012.
	\bibitem{Coho:iewu}
		Colleen Hoover,
		\emph{It Ends With Us},
		Atria Books, 2016.
\end{thebibliography}

\begin{table}[tbh]
	\begin{center}
		\begin{tabular}{llc} %{l|lr} create a vertical line %{ccr}inline to center %{llr} inline to right 
			\toprule %\hline %create a line 
			Name & Location & Height (m)\\ 
			\otoprule
			Arts Tower & Bolsover Street & 78\\
			St.~Georges Church & Broad Lane & 43\\
			Hicks Building & Hounsfield Road & 40\\
			\bottomrule\\ %rule make it have a bit gap 
		\end{tabular}
		\caption{The heights of some buildings at the University of Sheffield}  %put a caption below the table 
		\label{table:building-heights} %label the name of the table 
	\end{center}
\end{table}
From the Table~\ref{table:building-heights}, we can see blah blah blah.

\section{More commands}
\subsection{Text in maths mode}
Let 
\[
    \alpha = u +v\qquad \text{and}\qquad y = u - v. %quad make it have a space, qquad add extar spaces 
\]
\[
     \left(\frac{x}{y} + 1\right) %left right for correct size of bracket 
\]

\end{document} 

%homework 
\documentclass[11pt, a4paper]{amsart}
\usepackage{amsmath}
\usepackage{amsthm}
\usepackage{amssymb}
\usepackage{parskip}
\theoremstyle{definition}
\newtheorem{problem}{Problem}
\newenvironment{solution}
{\begin{proof}[Solution]}
{\end{proof}}


\title{MAS117: Homework 5}
\author{Nur Farhazni Najwa Binti Farhan}

\begin{document}

\maketitle

\begin{problem}
	Let $f_n$ denote the $n$\emph{-th of Fibonacci number}. Thus, $f_1 = 1$, $f_2 = 1$ and $f_{n+2} = f_{n+1} + f_n$ for all $n \in \mathbb{N}$. Use induction to show that	
	
\[
	f_1^2 + f_2^2 + \ldots +f_n^2 = f_n f_{n+1}
\]

for all natural numbers $n \in \mathbb{N}$.
\end{problem}

\begin{solution}
	The base case is $f_1^2 = f_1 f_2$. Since $f_1^2 = 1^2 = 1$, $f_1 f_{1+1} = f_1 f_2 = 1.1 = 1$, the base case is true.
	
	Now, we assume $f_1^2 + f_2^2 + \ldots +f_\alpha^2 = f_\alpha f_{\alpha+1}$ which $n = \alpha$. We need to prove $f_1^2 + f_2^2 + \ldots +f_\alpha^2 + f_{\alpha+1}^2 = f_{\alpha+1} f_{\alpha+2}$ which $n = \alpha+1$,

	$LHS = f_{\alpha} f_{\alpha+1} + f_{\alpha+1}^2 = f_{\alpha+1}(f_{\alpha} + f_{\alpha+1})$ by induction assumption. 

        Since $f_\alpha + f_{\alpha+1} = f_\alpha+2$, $LHS = f_{\alpha+1}(f_{\alpha} + f_{\alpha+1}) = f_{\alpha+1} f_{\alpha+2} = RHS$.

        Thus, 
        \[
	f_1^2 + f_2^2 + \ldots +f_n^2 = f_n f_{n+1}
\]
for all natural numbers $n \in \mathbb{N}$ is shown.
\end{solution}
\end{document} 
