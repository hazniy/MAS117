%homework 
\documentclass[11pt, a4paper]{amsart}
\usepackage{amsmath}
\usepackage{amsthm}
\usepackage{amssymb}
\usepackage{parskip}
\usepackage{listings}
\usepackage{pgfplots}
\lstset{
breaklines=true,
language=Python,
frame=single,
numbers=left,
showstringspaces=false,
}

\title{MAS117: Lab 6}
\author{Nur Farhazni Najwa Binti Farhan}

\begin{document} 
\maketitle
\section{Inserting computer code}
\subsection{The listings package}
\subsection{The verbatim environment}
The code below is an implementation of the 'Higher and Lower' game.
You start an enumerated list with \verb!\begin{enumerate}!
and end it with \verb+\end{enumerate}+
\begin{lstlisting}

# an implementation of the game of Higher or Lower

from random import randint

number = randint(1, 100)

print()
print("Higher or lower?")
print("----------------")
print()
print("Try to guess the number I've thought of between 1 and 100.")

attempts = 0
guess = 0

while guess != number:
    attempts += 1

    if attempts > 7:
        print("Hurry up!")

    attempt_number = str(attempts)
    guess = int(input("Guess "+attempt_number+": "))

    if guess > number:
        print(guess, "is too high")
    elif guess < number:
        print(guess, "is too low")

print("Correct! The answer was", guess)
print("You took", attempts, "attempts.")

if attempts > 8 and attempts < 11:
    print("Too slow for my liking!")
elif attempts > 10:
    print("Were you even trying?")
\end{lstlisting}
We have a variable \lstinline!height_max!.

\subsection{Including a file of code.}
Here is the source code of a file.
\lstset{
basicstyle=\ttfamily\footnotesize,
commentstyle=\color{brown},
keywordstyle=\color{red},
frame=none,
}
\lstinputlisting{higher_lower.py}

We define a function as follows:
\[
f(x) =
\begin{cases}
x^2 & \text{if } x \ge 0,\\
0 & \text{if } x < 0.
\end{cases}
\]

These are different: $3 - 4 = -1$,
double-barreled, pages~7--11.

The number before 0 is $-1$.
Their opening hours are 09:00 --17:00.

\begin{multline*}
first part of the expression \\
second part of the expression
\end{multline*}

\end{document} 
