\documentclass[11pt, a4paper]{amsart}
\usepackage{amsmath}
\usepackage{amsthm}
\usepackage{amssymb}
\usepackage{parskip}
\usepackage{listings}
\lstset{
breaklines=true,
language=Python,
frame=single,
numbers=left,
showstringspaces=false,
}

\title{Semester 1 Mini-project}
\author{240145846}

\begin{document}
\maketitle
\section{Introduction}
This project simulates a random walk by Annie the ant o a pentagon. Annie starts at vertex 0 and takes seven random steps, moving either clockwise or counterclockwise with equal probability.

\section{Objectives}
The goal is to simulate Annie's random walk with:
\begin{itemize}
  \item Fixed values for vertices (5) and steps (7).
  \item An extended version where the user specifies the number of vertices and steps.
\end{itemize}

\section{Python Code: Fixed Values}
\begin{lstlisting}[language=Python]
import random
from collections import Counter

#Define the number of vertices & steps for the journey
totvertices = 5 #Total vertices
totsteps = 7 #Number of steps

def randomwalk():
  position = 0 #Start position is always vertex 0
  for step in range(totsteps):
    move = random.choice([-1,1]) #Randomly choose the next step which -1(backward) or 1(forward)
    position = position + move #Update the current position

#Handle cases where the position becomes negative (which looping back to positive vertices)
  if position < 0:
    #Map (-) positions to the corresponding (+) vertices
    if position == -1 or position == -6:
      position = 4
    elif position == -2 or position == -7:
      position = 3
    elif position == -3:
      position = 2
    elif position == -4:
      position = 1
    elif position == -5:
      position = 0
  return position #Return the final position after the journey

finalpositions = [randomwalk() for i in range(128)] #Simulate 128 journeys (128 is chosen becuase it's 2^7, cont.
positioncounts = Counter(finalpositions)           #where 7 is the num of steps and 2 choices (backward & forward)

#Calculate and print percentages
print("\nvertex\tfrequency(%)")
print("-"* 26)
for vertex in range(totvertices):
    #Calculate the percentage for each vertex
    frequency = (positioncounts[vertex] / 128) * 100
    print(vertex, f"{frequency:.3f}", sep="\t")
\end{lstlisting}

Here, the program simulates 128 random walks and shows the frequency of each final vertex. The fixed values are 5 vertices and 7 steps.

\section{Extended Project: Adding User Prompts for Vertices and Steps}
The project has been extended to allow the user to specify both the number of vertices and the number of steps for the random walk. The updated code is shown below:

\begin{lstlisting}[language=Python]
import random
from collections import Counter

def randomwalk(vertices, steps):
    position = 0
    for i in range(steps):
        move = random.choice([-1, 1])
        position = position + move
    position = position % vertices #if vertices = 5: positions -1 and 5 both map to vertex 4. Always Positive
    return position

#Prompt the user for the number of steps
steps = int(input("Enter the number of steps: "))
vertices = int(input("Enter the number of vertices: "))
finalpositions = [randomwalk(vertices, steps) for i in range(pow(2, steps))]
positioncounts = Counter(finalpositions)

print("\nvertex\tfrequency (%)")
print("-" * 26)
for vertex in range(vertices):
    frequency = (positioncounts[vertex] / (pow(2, steps))) * 100
    print(vertex, f"{frequency:.3f}", sep="\t")
\end{lstlisting}

In this version, users input the number of vertices and steps. The program calculates the relative frequency of each final vertex.

\section{Effects of the Number of Steps and Vertices}
\subsection{Number of Steps}
More steps increase the likelihood of reaching different vertices.
For example:
\begin{itemize}
    \item With fewer steps (e.g., 3 steps), results are concentrated near the starting point.
    \item With more steps (e.g., 10 steps), resultrs are more spread out.
\end{itemize}

\subsection{Number of Vertices}
Increasing the vertices spreads the distribution.
For example:
\begin{itemize}
    \item With fewer vertices (e.g., 3 vertices), the final positions are concentrated.
    \item With more vertices (e.g., 10 vertices), the final positions are more varied.
\end{itemize}

The number of steps and vertices both play a big role in shaping where Annie ends up. With more vertices, her final positions are spread out across a larger range. Similarly, taking more steps increases the chances of her exploring more of the vertices along the way.

\end{document}
